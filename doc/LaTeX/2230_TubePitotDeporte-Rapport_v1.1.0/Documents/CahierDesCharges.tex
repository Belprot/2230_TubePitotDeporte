Le but du projet consiste à développer un système permettant de détecter l’angle d’incidence et la vitesse au décrochage d’un avion. La fixation de ce système doit être flexible afin de pouvoir l’installer sur différents types d’avions. L’emplacement de fixation ne doit pas se trouver dans le flux d’air provenant de l’hélice afin d’éviter que la mesure de vitesse ne soit faussée. Il doit également être miniaturisé au maximum afin de produire le minimum de traînée possible et de ne pas dépasser un poids de 500g. Les données acquises par les capteurs doivent être transmises de la partie déportée à un appareil Android se trouvant dans le cockpit de l’avion au travers une communication sans fil. L’appareil Android doit traiter et afficher les données reçues, si possible graphiquement (optionnel). \\

\noindent Le cahier des charges complet se trouve en annexes.
\vspace{1 cm}

